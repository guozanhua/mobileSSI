%%%%%%%%%%%%%%%%%%%%%%%%%%%%%%%%%%%%%%%%%
% Arsclassica Article
% LaTeX Template
% Version 1.1 (10/6/14)
%
% This template has been downloaded from:
% http://www.LaTeXTemplates.com
%
% Original author:
% Lorenzo Pantieri (http://www.lorenzopantieri.net) with extensive modifications by:
% Vel (vel@latextemplates.com)
%
% License:
% CC BY-NC-SA 3.0 (http://creativecommons.org/licenses/by-nc-sa/3.0/)
%
%%%%%%%%%%%%%%%%%%%%%%%%%%%%%%%%%%%%%%%%%

%----------------------------------------------------------------------------------------
%	PACKAGES AND OTHER DOCUMENT CONFIGURATIONS
%----------------------------------------------------------------------------------------

\documentclass[
10pt, % Main document font size
a4paper, % Paper type, use 'letterpaper' for US Letter paper
oneside, % One page layout (no page indentation)
%twoside, % Two page layout (page indentation for binding and different headers)
headinclude,footinclude, % Extra spacing for the header and footer
BCOR5mm, % Binding correction
]{scrartcl}

%%%%%%%%%%%%%%%%%%%%%%%%%%%%%%%%%%%%%%%%%
% Arsclassica Article
% Structure Specification File
%
% This file has been downloaded from:
% http://www.LaTeXTemplates.com
%
% Original author:
% Lorenzo Pantieri (http://www.lorenzopantieri.net) with extensive modifications by:
% Vel (vel@latextemplates.com)
%
% License:
% CC BY-NC-SA 3.0 (http://creativecommons.org/licenses/by-nc-sa/3.0/)
%
%%%%%%%%%%%%%%%%%%%%%%%%%%%%%%%%%%%%%%%%%

%----------------------------------------------------------------------------------------
%	REQUIRED PACKAGES
%----------------------------------------------------------------------------------------

\usepackage[
nochapters, % Turn off chapters since this is an article        
beramono, % Use the Bera Mono font for monospaced text (\texttt)
eulermath,% Use the Euler font for mathematics
pdfspacing, % Makes use of pdftex’ letter spacing capabilities via the microtype package
dottedtoc % Dotted lines leading to the page numbers in the table of contents
]{classicthesis} % The layout is based on the Classic Thesis style

\usepackage{arsclassica} % Modifies the Classic Thesis package

\usepackage[T1]{fontenc} % Use 8-bit encoding that has 256 glyphs

\usepackage[utf8]{inputenc} % Required for including letters with accents

\usepackage{graphicx} % Required for including images
\graphicspath{{Figures/}} % Set the default folder for images

\usepackage{enumitem} % Required for manipulating the whitespace between and within lists

\usepackage{lipsum} % Used for inserting dummy 'Lorem ipsum' text into the template

\usepackage{subfig} % Required for creating figures with multiple parts (subfigures)

\usepackage{amsmath,amssymb,amsthm} % For including math equations, theorems, symbols, etc

\usepackage{varioref} % More descriptive referencing

%----------------------------------------------------------------------------------------
%	THEOREM STYLES
%---------------------------------------------------------------------------------------

\theoremstyle{definition} % Define theorem styles here based on the definition style (used for definitions and examples)
\newtheorem{definition}{Definition}

\theoremstyle{plain} % Define theorem styles here based on the plain style (used for theorems, lemmas, propositions)
\newtheorem{theorem}{Theorem}

\theoremstyle{remark} % Define theorem styles here based on the remark style (used for remarks and notes)

%----------------------------------------------------------------------------------------
%	HYPERLINKS
%---------------------------------------------------------------------------------------

\hypersetup{
%draft, % Uncomment to remove all links (useful for printing in black and white)
colorlinks=true, breaklinks=true, bookmarks=true,bookmarksnumbered,
urlcolor=webbrown, linkcolor=RoyalBlue, citecolor=webgreen, % Link colors
pdftitle={}, % PDF title
pdfauthor={\textcopyright}, % PDF Author
pdfsubject={}, % PDF Subject
pdfkeywords={}, % PDF Keywords
pdfcreator={pdfLaTeX}, % PDF Creator
pdfproducer={LaTeX with hyperref and ClassicThesis} % PDF producer
} % Include the structure.tex file which specified the document structure and layout
\usepackage{listings}
\usepackage{color}
\usepackage[utf8]{inputenc} 

\definecolor{mygreen}{rgb}{0,0.6,0}
\definecolor{mygray}{rgb}{0.5,0.5,0.5}
\definecolor{mymauve}{rgb}{0.58,0,0.2}


\lstset{ %
  backgroundcolor=\color{white},   % choose the background color; you must add \usepackage{color} or \usepackage{xcolor}
  basicstyle=\footnotesize,        % the size of the fonts that are used for the code
  breakatwhitespace=false,         % sets if automatic breaks should only happen at whitespace
  breaklines=true,                 % sets automatic line breaking
  captionpos=b,                    % sets the caption-position to bottom
  commentstyle=\color{mygreen},    % comment style
  deletekeywords={...},            % if you want to delete keywords from the given language
  escapeinside={\%*}{*)},          % if you want to add LaTeX within your code
  extendedchars=true,              % lets you use non-ASCII characters; for 8-bits encodings only, does not work with UTF-8
 % frame=single,                    % adds a frame around the code
  keepspaces=true,                 % keeps spaces in text, useful for keeping indentation of code (possibly needs columns=flexible)
  keywordstyle=\color{blue},       % keyword style
  language=Octave,                 % the language of the code
  morekeywords={*,...},            % if you want to add more keywords to the set
  numbers=left,                    % where to put the line-numbers; possible values are (none, left, right)
  numbersep=5pt,                   % how far the line-numbers are from the code
  numberstyle=\tiny\color{mygray}, % the style that is used for the line-numbers
  rulecolor=\color{black},         % if not set, the frame-color may be changed on line-breaks within not-black text (e.g. comments (green here))
  showspaces=false,                % show spaces everywhere adding particular underscores; it overrides 'showstringspaces'
  showstringspaces=false,          % underline spaces within strings only
  showtabs=false,                  % show tabs within strings adding particular underscores
  stepnumber=2,                    % the step between two line-numbers. If it's 1, each line will be numbered
  stringstyle=\color{mymauve},     % string literal style
  tabsize=2,                       % sets default tabsize to 2 spaces
  title=\lstname                   % show the filename of files included with \lstinputlisting; also try caption instead of title
}
\hyphenation{Fortran hy-phen-ation} % Specify custom hyphenation points in words with dashes where you would like hyphenation to occur, or alternatively, don't put any dashes in a word to stop hyphenation altogether

%----------------------------------------------------------------------------------------
%	TITLE AND AUTHOR(S)
%-----------------------------------------------------------------<WorkloadPage  config="Quantitaetstheorie.xml" rows="10" colums="1" stressLevel="2" isStressfull="true" pageTimeout="180" url="Quantitätstheorie.htm" />-----------------------

\title{\normalfont\spacedallcaps{SSI Linux \&Android Port Log}} % The article title

\author{\spacedlowsmallcaps{Simon Flutura}} % The article author(s) - author affiliations need to be specified in the AUTHOR AFFILIATIONS block

\date{} % An optional date to appear under the author(s)

%----------------------------------------------------------------------------------------

\begin{document}

\lstset{language=bash}  
%----------------------------------------------------------------------------------------
%	HEADERS
%----------------------------------------------------------------------------------------

\renewcommand{\sectionmark}[1]{\markright{\spacedlowsmallcaps{#1}}} % The header for all pages (oneside) or for even pages (twoside)
%\renewcommand{\subsectionmark}[1]{\markright{\thesubsection~#1}} % Uncomment when using the twoside option - this modifies the header on odd pages
\lehead{\mbox{\llap{\small\thepage\kern1em\color{halfgray} \vline}\color{halfgray}\hspace{0.5em}\rightmark\hfil}} % The header style

\pagestyle{scrheadings} % Enable the headers specified in this block

%----------------------------------------------------------------------------------------
%	TABLE OF CONTENTS & LISTS OF FIGURES AND TABLES
%----------------------------------------------------------------------------------------

\maketitle % Print the title/author/date block

\setcounter{tocdepth}{2} % Set the depth of the table of contents to show sections and subsections only

\tableofcontents % Print the table of contents

% \listoffigures % Print the list of figures

% \listoftables % Print the list of tables

%----------------------------------------------------------------------------------------
%	ABSTRACT
%----------------------------------------------------------------------------------------

 % \section*{Abstract} % This section will not appear in the table of contents due to the star (\section*)

 % Dummy text

%----------------------------------------------------------------------------------------
%	AUTHOR AFFILIATIONS
%----------------------------------------------------------------------------------------



%----------------------------------------------------------------------------------------

\newpage % Start the article content on the second page, remove this if you have a longer abstract that goes onto the second page

%----------------------------------------------------------------------------------------
%	Mobile SSI
%----------------------------------------------------------------------------------------


\section{Ported Plugins and Notes}

\subsection{ssiframe}

\subsection{ssievent}

\subsection{ssiioput}

\subsection{ssimouse}
This plugin uses libxinput2 on linux, thus depends on xorg.
Not ported to Android!

\subsection{ssiaudio}
This plugin depends on portaudio on linux and

OpenSE on android.
\subsection{ssimodel}

\subsection{ssisignal}

\subsection{ssiemovoice}

\subsection{ssiopensmile}
not tested

\subsection{ssiandroidsensors}

Plugin only avaliable on Android.
Enables messages via logcat.

\verb|ssimsg=new ssi::AndroidMessage();|

\subsection{ssiandroidjavasensors}
Android only. Sends java events to ssi. E.g bluetooth or battery.
\subsection{ssiwebsockets}


\subsection{ssivectorfusion}

\subsection{ssiffmpeg}
not tested


\section{SSJ integration}

for ssj integration a ssj sensor plugin is added to ssi;
int is wrapped on ssj side via sensorProviderSSI, that replaces ssj core with ssi.

other posibilities include running both frameworks in parallel wher an ssj consumer is paired to the ssi ssj sensor.

following JNI snipplets might be helpful:

\section{APK Build Support}

see android doc for build instructions.
SSI uses android.apk.cmake to create an APK android package.
ANT is used; todo switch to maven?


At the moment the APK is build around \verb|android_xmlpipe| found in
\verb|plugins/androidSensors/tools/xmlpipe|.
The java code found in \verb|docs/apk| is used to extract native libraries then the android main activity in \verb|android_xmlpipe| is started.
After extracting the pipeline using native assetmanager, xmlpipe is started as usual.

\section{SSI Cmake Buildsystem}

Old Visual Studio Buildsystem is untouched,
but Visual Studio Projects can be generated from Cmake.

Cmake generates CodeBlocks make projects, linux make projects.



\subsection{Structure}

Each project has to live in its own subdirectory for better cmake integration.
Each subproject has its own CMakeLists.txt file.
Every library or executeable is a subproject of its own to keep things simple.

A directory containing CMake-projects has to contain a CMakeLists.txt mentioning subpojects.

Variables for SSI install path, \verb|plattform/compiler| detection etc can be found in the \verb|trunk/CMakeLists.txt|


\subsection{how to add plugin source}

Add subdirectory to \verb|trunk/plugins/CMakeLists.txt|
with \verb|add_subdirectory(dirname)|.

Add a CMakeLists.txt:

\label{subsec:how to add a plugin}
\begin{lstlisting}[caption={cmake ssi plugin}]
# add project
project(ssimouse)

#add subdirectory for text
add_subdirectory(test)
#add tests dependencies
add_dependencies(ssimouse_test ssi ssimouse)


#add include directories
include_directories (
	include
	../../core/include
	../../core/include/ioput/socket
	../../core/include/ioput
	
	../event/include
	
	../frame/include
	
	../graphic/include
	
	../ioput/include
	../ioput/include/ioput/socket
	../ioput/include/ioput
	../
)

#set source files
set(COMMON_SRC_FILES

 source/ExportMain.cpp
 source/Mouse.cpp
 source/CursorMover.cpp
 
)

#find librarys

IF(MINGW)
  find_library (MINGW_WSOCKET "wsock32" HINTS ${MINGWLIB} )
  find_library (MINGW_WMM "winmm" HINTS ${MINGWLIB})
  find_library (MINGW_WSOCKET2 "ws2_32" HINTS ${MINGWLIB} )
  find_library (MINGW_PTHREAD "pthread" HINTS ${MINGWLIB} )
  #set compiler flags for c++11 threading and debug
  #todo create gcc/make debug target?
  	set( CMAKE_CXX_FLAGS "${CMAKE_CXX_FLAGS} -std=c++11 -ggdb" )
  	
  ELSEIF(UNIX)
  SET(MINGW_WSOCKET "")
  SET(MINGW_WSOCKET2 "")
  SET(MINGW_WMM "")
  
  set( CMAKE_CXX_FLAGS "${CMAKE_CXX_FLAGS} -std=c++11 -ggdb" )
  
  find_library (MINGW_PTHREAD "pthread" HINTS "/usr/lib" )
  # x11 dependencies
  find_library (X11 "X11" HINTS "/usr/X11R6/lib")
  find_library (Xi "Xi" HINTS "/usr/X11R6/lib")

ELSE(MINGW) # visual studio has its own ways for winsockets/threading eg #pragma
  SET(MINGW_WSOCKET "")
  SET(MINGW_WSOCKET2 "")
  SET(MINGW_WMM "")
   SET(MINGW_PTHREAD "")
ENDIF(MINGW)


set(SRC_FILES ${COMMON_SRC_FILES} )

# add main target
add_library(ssimouse SHARED ${SRC_FILES})

# link against internal and external librarys
IF(UNIX)
target_link_libraries(ssimouse ssi ${MINGW_WSOCKET} ${MINGW_WSOCKET2} ${MINGW_WMM} ${MINGW_PTHREAD})
ELSE(UNIX)
target_link_libraries(ssimouse ssi ${MINGW_WSOCKET} ${MINGW_WSOCKET2} ${MINGW_WMM} ${MINGW_PTHREAD} ${X11} ${Xi})
ENDIF(UNIX)

#add_executable(myapp main.c)

#rename targets if debug
set_target_properties(ssimouse PROPERTIES DEBUG_POSTFIX "d" PREFIX "")

#install target to ssi install path set in base directory
install(TARGETS ssimouse DESTINATION ${SSI_INSTALL}/${SSI_PLATFORM}/${SSI_COMPILER})
\end{lstlisting}

\subsection{how to add a plugin library}

The library has to be added to the ssi directory tree in trunk/libs/shared/libname.
Binarys have to be added into according plattform and compiler subdirs.
The library should contain a macro for easier use in projects, if it contains multiple files.

Thereafter the library can be added to the project:



\label{subsec:how to add a plugin library}
\begin{lstlisting}[caption={cmake ssi plugins library}]
# add project
project(ssishore)

#add test
add_subdirectory(test)
add_dependencies(ssishore_test ssi ssishore)

# add include dirs
include_directories (
	include
	../../core/include
	../../libs/shared/opencv/include/
	../../plugins
)

# set source files
set(COMMON_SRC_FILES

source/ExportMain.cpp

)


set(SRC_FILES ${COMMON_SRC_FILES} )


# set libraries libs path
get_filename_component(OPENCV_PATH ../../libs/shared/opencv/libs/${SSI_PLATFORM}/${SSI_COMPILER} ABSOLUTE)
# set libraries bin path
get_filename_component(OPENCV_PATH_SHARED ../../libs/shared/opencv/bin/${SSI_PLATFORM}/${SSI_COMPILER} ABSOLUTE)


# use macro to find precompiled opencv libs
include(../../libs/shared/opencv/opencv_paths.cmake)

opencvPaths(${OPENCV_PATH} ${OPENCV_PATH_SHARED} OPENCV_LIB_DEBUG OPENCV_LIB OPENCV_SHARED_DEBUG OPENCV_SHARED)

					
# find single file static library for linking
find_library(
			SHORE_LIB
			
			NAMES
				
				shore140.lib		 

			HINTS
				../../libs/shared/shore/libs/${SSI_PLATFORM}/${SSI_COMPILER}
			
			)

# find single file dynamic library for copiing
find_file(
			SHORE_SHARED
			
			NAMES
				
				shore140.dll		 

			HINTS
				../../libs/shared/shore/bin/${SSI_PLATFORM}/${SSI_COMPILER}
			
			)

# add main target

add_library(ssishore SHARED ${SRC_FILES})


# link for debug
target_link_libraries(ssishore debug ssi
													${OPENCV_LIB_DEBUG} ${SHORE_LIB}
													)

# link for release
target_link_libraries(ssishore optimized ssi
													${OPENCV_LIB} ${SHORE_LIB}
													)
#add_executable(myapp main.c)

#rename targets if debug
set_target_properties(ssishore PROPERTIES DEBUG_POSTFIX "d")

#install target to ssi install path set in base directory
install(TARGETS ssishore DESTINATION ${SSI_INSTALL}/${SSI_PLATFORM}/${SSI_COMPILER})

#install dll files (copy)
install(FILES ${SHORE_SHARED} ${OPENCV_SHARED} ${OPENCV_SHARED_DEBUG} DESTINATION ${SSI_INSTALL}/${SSI_PLATFORM}/${SSI_COMPILER})
\end{lstlisting}

Here follows the macro for adding multiple lib files:

\label{subsec:how to add a plugin library}
\begin{lstlisting}[caption={cmake library macro}]

function( opencvPaths OPENCV_PATH OPENCV_PATH_SHARED OPENCV_LIB_DEBUG OPENCV_LIB OPENCV_SHARED_DEBUG OPENCV_SHARED) 

	# add static debug libs
	if(WIN32)			
		set(${OPENCV_LIB_DEBUG}
							
								${OPENCV_PATH}/opencv_core2410d.lib
								${OPENCV_PATH}/opencv_features2d2410d.lib	

						PARENT_SCOPE)
	endif(WIN32)

	# add static release libs
	if(WIN32)
		set(${OPENCV_LIB}
							
								${OPENCV_PATH}/opencv_core2410.lib
								${OPENCV_PATH}/opencv_features2d2410.lib

								PARENT_SCOPE)
	endif(WIN32)
	
	# add debug dlls
	if(WIN32)			
		set(${OPENCV_SHARED_DEBUG}
							
								${OPENCV_PATH_SHARED}/opencv_core2410d.dll
								${OPENCV_PATH_SHARED}/opencv_features2d2410d.dll	
	
							
								PARENT_SCOPE)
	endif(WIN32)

	# add release dlls
	if(WIN32)
		set(${OPENCV_SHARED}
							
								${OPENCV_PATH_SHARED}/opencv_core2410.dll
								${OPENCV_PATH_SHARED}/opencv_features2d2410.dll

								PARENT_SCOPE)
	endif(WIN32)


endfunction(opencvPaths)
\end{lstlisting}



\subsection{how to add a source file to Core}
Simply add the file to the \verb|COMMON_SRC_FILES| list in trunk/core/CMakeList.txt.
If the file contains code that is only used on one plattform, add it to \verb|P_SRC_FILES|
in the according case.

\label{subsec:how to add a source file to Core}
\begin{lstlisting}[caption={add source file to core}]

set(COMMON_SRC_FILES

source/buffer/Buffer.cpp

)
 IF(WIN32)
 set(P_SRC_FILES
	source/ioput/socket/ip/win32/NetworkingUtils.cpp
	)
 ELSE(WIN32)
 set(P_SRC_FILES
	source/ioput/socket/ip/posix/NetworkingUtils.cpp

	
	)
 ENDIF(WIN32)
\end{lstlisting}

\subsection{how to add a test or tool}

Tests are a subproject of their own.
They need to be added to the parent directory.


\label{subsec:how to add a test or tool}
\begin{lstlisting}[caption={cmake subproject}]
# add project
project(ssiframe_test)

# add incluede dirs
include_directories (
	.
	../../../core/include
	../../../core/include/ioput/socket
	../../../core/include/ioput
	
	../../../plugins/
)

# add source files
set(COMMON_SRC_FILES

 Main_.cpp
)


set(SRC_FILES ${COMMON_SRC_FILES} )

#find librarys

IF(MINGW)
  find_library (MINGW_WSOCKET "wsock32" HINTS ${MINGWLIB} )
  find_library (MINGW_WMM "winmm" HINTS ${MINGWLIB})
  find_library (MINGW_WSOCKET2 "ws2_32" HINTS ${MINGWLIB} )
  find_library (MINGW_PTHREAD "pthread" HINTS ${MINGWLIB} )
  #set compiler flags for c++11 and debug
  #todo debug target on gcc
  	set( CMAKE_CXX_FLAGS "${CMAKE_CXX_FLAGS} -std=c++11 -ggdb" )
  	
  ELSEIF(UNIX)
  SET(MINGW_WSOCKET "")
  SET(MINGW_WSOCKET2 "")
  SET(MINGW_WMM "")
  set( CMAKE_CXX_FLAGS "${CMAKE_CXX_FLAGS} -std=c++11 -ggdb" )
  find_library (MINGW_PTHREAD "pthread" HINTS "/usr/lib" )
  find_library (X11 "X11" HINTS "/usr/X11R6/lib")
  find_library (Xi "Xi" HINTS "/usr/X11R6/lib")
  
ELSE(MINGW)
  SET(MINGW_WSOCKET "")
  SET(MINGW_WSOCKET2 "")
  SET(MINGW_WMM "")
   SET(MINGW_PTHREAD "")
ENDIF(MINGW)

# add projects main target (library or binary)
add_executable(ssiframe_test ${SRC_FILES})

# add linking to other projects or external librarys
target_link_libraries(ssiframe_test ssi ${MINGW_WSOCKET} ${MINGW_WMM} ${MINGW_WSOCKET2} ${MINGW_PTHREAD})


#rename targets if debug
set_target_properties(ssiframe_test PROPERTIES DEBUG_POSTFIX "d")

#install target to ssi install path set in base directory
install(TARGETS ssiframe_test DESTINATION ${SSI_INSTALL}/${SSI_PLATFORM}/${SSI_COMPILER})
\end{lstlisting}

\subsection{compiler parameters}
for gcc c++11 threading

\subsection{doxygen}
\verb|http://majewsky.wordpress.com/2010/08/14/tip-of-the-day-cmake-and-doxygen/|
\subsection{tests}

\verb|http://mifrosu.blogspot.de/2013/02/cmake-and-google-test-framework.html|


\section{Threading}
Cx11 on Android.
Just depends on version of gcc compiler(4.8+).

\subsection{c++11}
std::thread
contains conditional variables

problem: interruptable threads

\subsection{windows}
old ssi threading system is untouched

\subsection{posix}
are implemented via c++11
interruptable threads can be implemented via pthread-cancle and pthread-kill


\section{timer}
Timers need a plattform independent abstraction.
Windows has its own timers, linux uses Posix standard, android features its own challanges as suspend disturbs the timers.

boost.timer?

\verb|std::chrono::high resolution clock|?

\verb|http://stackoverflow.com/questions/1487695/c-cross-platform-high-resolution-timer|

linux monolitic raw

\section{sockets}
on linux udp sockets might only be able to send, when their data is recived due to icmp packages.

\section{file tools}
file tools for selecting all files in subdir and especially file dialogs have to be ported.

\section{named pipes}
named pipes are not yet ported.

\section{decentralization \& syncronization?}

already features for streaming ssi info ssi.

synchronisation using ms since 1970?
\verb|std::chrono::time_point::time_since_epoch|

needs 64 bit:

	\verb|uint32_t| ms overflow after 50 days
	\verb|uint64_t| ms overflow after 584942417 years

\section{GUI rewrite}

The plotting part needs to be rewritten.
\subsection{sdl2}
multiwindow since 2.0.
has support on many plattforms, win, linux, android, osx.

no support for multithreaded rendering:
use cairo for rendering, sdl2 to display

sdl windowmanager runs its own thread, all sdl calls should happen there.
\subsection{windowmanagment linux}
multi window and (multi)console handling

\subsection{qt}
big dependency.

\subsection{html5 \& websockets}

\section{Building SSI for Android}

Note: this needs Android-NDK set up on your system.
follow instructions from /docs/ssi-port-cmake/intro-android(-from-win)

\section{Crosscompilation}
\begin{enumerate}
\item crosscompile linux -> arm linux (android)
see android and win-android documentation for an up to date tutorial.


\item linux -> win32/64 mingw


crosscompilation via mxe:
\verb|cmake ../trunk/ -DCMAKE_TOOLCHAIN_FILE=~/mxe/usr/i686-w64-mingw32/share/cmake/mxe-conf.cmake|

\item windows -> arm linux (android)?

\end{enumerate}



\section{Buildsystems overview}

A main problem of making a project portable is choosing the right buildsystem.
The SSI requirements would be integration into multiple IDEs as Visual Studio and Code::Blocks as well as crossplattform support with windows, android and Linux support and
crosscompiling eg Linux to Android.

\subsection{Cmake}
biggest community.
fullfills requirements.
Module scripts necessary.
Bad automatisation, needs to be adjusted per hand, hard to update cmake from visual studio.
\subsection{Gyp}
Googles Buildsystem might be an option

\subsection{Waf}

Written in python, adjustable.
Fullfills all requirements but is not as widely used as cmake.
Also suffers from bad integration into IDEs.

\subsection{Jam and others}

Projects such as Boost use jam for (cross plattform) building.
Many different versions therefore fragmented community.

Same goes for several other buildsystems.

\section{Bugs worth mentioning}

\subsection{Singleton reinitialization on Android}
ssi factory does not get destroyed when backgroundservice is stopped.
constructor is not called, structs have to be reinitialized manually!

\subsection{Memory leak in Linux mouse-plugin}

free memory allocated by x11 function XIQueryPointer() manually.
\verb|free(buttons.mask);|

\subsection{Memory corruption in Linux SignalPainter}

included wrong header.
memory got allocated using new compiled symbol but old struct was expected.

\subsection{Mutex violation with c++11 condition variables}

c++11 condition variables use std::locks instead of mutexes.
creating locks on the fly leads to opening the mutex in the wrong place.

use \verb|condition_variable_any| instead.


\subsection{UDP Broadcast not working}

setEnableBroadcast socket_opt was not set in send but only in send to

\subsection{Memory corruption in Linux EventMonitor}


%----------------------------------------------------------------------------------------
%	BIBLIOGRAPHY
%----------------------------------------------------------------------------------------

\renewcommand{\refname}{\spacedlowsmallcaps{References}} % For modifying the bibliography heading

\bibliographystyle{unsrt}

\bibliography{sample.bib} % The file containing the bibliography

%----------------------------------------------------------------------------------------

\end{document}
